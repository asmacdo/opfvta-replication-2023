\section{Abstract}

The value of increasingly many research articles is inextricably contingent on data analysis results which substantiate their claims.
Unlike data production steps, data analysis steps lend themselves to a higher standard of both transparency and repeated operator-free execution.
This is accomplished via fully reexecutable research outputs, which contain the entire instruction set for end-to-end generation of an entire article solely from the earliest feasible provenance point, in a programatically executable format.
In this study, we make use of a peer-reviewed neuroimaging article, which provides rudimentary capabilities, in order to formulate a new reexecution model.
We detail the issues with rendering reexecution accessible, and specify a number of best practices which permitted us to generate a new reexecution model mitigating most of the encountered issues.
This new model improves the reliability and portability of reexecution, and permits the production and tracking of numerous reexecution outputs.
We further show how these capabilities can subsequently be used in order to provide reproducibility assesments, both via simple statistical metrics, and by visually highlighting divergent elements for human inspection.
We argue that reexecutable articles are thus a feasible best practice, the usage of which can greatly enhance the understanding of data analysis variability.
We encourage re-use and derivation of the model produced herein, which is rendered modular as a matter of design, and in which reexecutable article code, data, and environment specifications can be easily substituted or adapted.

% Maybe also mention, though ideally somewhere else:
% * The comparison article we thus produce, i.e. this article, is similarly reexecutable.
% * ... a collection of best practices, including YODA principles for data management, containers for preservation, Gentoo for long term flexibility.
% * This is a very good point, though again, maybe not for the abstract “The inability to re-use code due to instability thus limits the lasting value of the work as a repository of procedural knowledge”.
% *We document a number of prominent difficulties with de novo article generation, arising from the rapid evolution of extrinsic tools, and from nondeterministic data analysis procedures.

