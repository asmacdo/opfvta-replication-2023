\section{Abstract}

The value of research articles is increasingly contingent on data analysis results which substantiate their claims.
Unlike data production steps, data analysis steps lend themselves to a higher standard of both transparency and repeated operator-free execution.
This higher standard can be approached via fully reexecutable research outputs, which contain the entire instruction set for end-to-end generation of an entire article solely from the earliest feasible provenance point, in a programatically executable format.
In this study, we make use of a peer-reviewed neuroimaging article, which provides complete but fragile reexecution instructions, as a starting point to formulate a new reexecution model which is both robust and portable.
We render this model modular as a core design aspect, so that reexecutable article code, data, and environment specifications can be easily substituted or adapted.
In conjunction with this model, which forms the main reusable product of this study, we detail the core challenges with full article reexecution and specify a number of best practices which permitted us to mitigate them.
We further show how the capabilities of our model can subsequently be used to provide reproducibility assesments, both via simple statistical metrics and by visually highlighting divergent elements for human inspection.
We argue that reexecutable articles are thus a feasible best practice, the usage of which can greatly enhance the understanding of data analysis variability.
Lastly, we comment at length on the outlook for reexecutable resource and encourage re-use and derivation of the model produced herein.

% Maybe also mention, though ideally somewhere else:
% * The comparison article we thus produce, i.e. this article, is similarly reexecutable.
% * ... a collection of best practices, including YODA principles for data management, containers for preservation, Gentoo for long term flexibility.
% * This is a very good point, though again, maybe not for the abstract “The inability to re-use code due to instability thus limits the lasting value of the work as a repository of procedural knowledge”.
% *We document a number of prominent difficulties with de novo article generation, arising from the rapid evolution of extrinsic tools, and from nondeterministic data analysis procedures.

