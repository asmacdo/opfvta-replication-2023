\section{Discussion}

In this article we present an automated workflow for full article reexecution.
This first effort of its kind not only substantiates the fundamental relevance of reexecution as a process, but also provides a reference implementation of technologies required to provide this capability.
Further, we detail important and transferable principles, and document the manifold intricacies of creating a reexecution workflow.

\subsection{Reexecutability}
% this should be softened up.
We argue that if qualitative statements cannot be underpinned by statistical summaries re-generated from the self-same same data, concerns such as reproducibility in view of novel data are rendered immaterial.
Further, we argue that if qualitative statements are generated manually, they are subject to the same potential confounds as any original resource, and even perhaps biased by it.
In effect, automatic reexecutability is a prerequisite for any claims on broader reproducibility.

We demonstrate the feasibility of full research output reexecution given cutting-edge technologies, and publish all resources for open access, adaptation, inspection, and re-use.
The resource topology which we produce isolates data and original resources, and does not make assumptions about the internal structure of a reexecutable article, and is of course, not domain-specific.
In particular, the code of the original article, is managed as a sumbodule, and triggered via a freely configurable and lightweight Bash script.

Our topology also provides full support for version tracking all required resources, thus offering perfect transparency into what instructions and which external data is used in article reexecution.
In particular, we rely heavily on DataLad, a technology which renders the versioning capabilities of Git usable in the context of large binary data.

We also provide a template for evaluating reproduction attempts, based on the usage of page-wise difference plotting, and presenting excerpts with a high density of differences for visual inspection.
In light of this comparison, make an argument for the value of \textit{article} reexecutability.
As it is the article which is the accepted research output format, any external subset of values which would be earmarked as outputs-of-choice for reexecution risk both overestimation as well as underestimation of relevant differences.
For example, if large differences in top-level statistical data are observed, but bear no relevance to the statements made in the article, these differences have little bearing on any analysis of the reexecution quality.
By contrast, cherry-picking all of the commonly very numerous inline statistical summaries (F- or p-values) and correctly linking them to the respective statements they underpin in an extraneous format can be anywhere from error-prone to infeasible.
Thus, very relevant differences may go undetected, or be difficult to assign a context to.

However, we concede that more advanced output formats may emerge as research reexecutability becomes a more widespread capability and concern, and that top-level output data reexecution may then offer considerable advantages.
We speculate that these may lie in the field of not just determining whether statements made are unreliable, but further, if valuable statements remained untapped by the original researchers due to simple computational unreliability.
For this purpose our workflow also produces and records all of the top-level data (statistical maps, data tables, etc.) from which the article extracts elements relevant to its statements.

\subsection{Reproducibility}

We sharply distinguish between reexecutability, and reproducibility.
The former refers to the capability of an analogue research output (with any or even no consistency) to be regenerated exclusively from the earliest feasible data provenance and from automatically executable instructions used to generate the original output.
The latter refers to the quality of the reexecution with respect to supporting the original research output claims.
We further distinguish reproducibility — which may be good or poor based on the expected coherence standard — from numerically identical reproduction of statistical metrics, which we deem as a distinct quality, termed replicability.

Reexecutability is an indispensable prerequisite for reproducibility assessment in any computational context.
We supplement the reexecution workflow description of this article with a brief demonstration of how it can be used to provide a reproducibility assesment.
For this purpose we use a difference computation tool (in computational contexts known simply as “diff”) which summarizes and visually displays mismatches between a historical manuscript record and multiple reexecutions over various environments.
Such a process makes mismatches visible at-a-glance throughout the article text, rendering them them easy to locate and interpret via human inspection.

Based on these results we lay out a few key findings for further reproducibility assessment given the level of stringency which full-article comparison introduces.
In particular, we notice that figures which map output data directly are highly — and to a consistent extent — variable across multiple reexecution attempts.
However, in as far as such figures are accompanied by statistical evaluations, we find these to be highly consistent.
What is and is not consistent is of course contingent on the reexecuted article itself, but we find this phenomenon of notable relevance.
It indicates that reproduction quality is not only reliant on whether or not data processing is deterministic, but also on which aspects of the top-level data the authors seek to highlight.

In neuroimaging workflows, the most notorious source for non-determinacy in data analysis is the registration, which operates via a random starting point — specified by a seed value — and iterates according to a gradient descent algorithm from there.
While the toolkit used by the article reexecuted here allows the specification of a particular seed, this is not a feature commonly used by operators.
In light of our results the question emerges whether or not this is indeed a good practice.
It stands to reason that a specific seed — whether by coincidence or \textit{ex post facto} selection — may result in an anomalous conclusion.
It may then be that a stronger finding is one which is robust with respect to the preprocessing variability, even if this comes at the cost of compromising bit-identical replicability.
Conversely, it could be argued that reproduction analysis can be better targeted and more concise, if seed values were fixed to universally accepted numbers (analogous to the usage of nothing-up-my-sleeve numbers in cryptography).

\subsection{Challenges}
For this meta-article we have selected an original neuroimaging article which already strove to encapsulate all of the instructions needed to reproduce itself in its entirety simply from raw data and automatically executable instructions.
Even in light of this uncommon advantage, ensuring reexecutability has proven an effort-intensive process.
Difficulties arose primarily due to the software stack instability.
It is common for scientific software — and more and more common as one moves towards the cutting edge — to receive frequent interface changes, and to drop support for older dependencies.

In this article we propose container technology as a mitigation method for such instability, however, these are not without draw-backs, as they can make introspection more challenging.
In view of this, we distribute interactive container targets, whereby the user may enter the container dedicated to automatic reexecution manually, in order to be able to inspect and test changes in the environment.
Even so, on account of these containers being dedicated to automatic execution, features such as an advanced text processor are missing, and the inclusion of such features may not be ultimately desired.

%\subsection{Caveats}
% Do we want this to be different from challenges?

\subsection{Outlook}

We propose a few key considerations for the further development of the present technology, albeit also noting that practical reuse might better identify promising enhancements better than theoretical considerations.

In particular, we find that reexecutable article debugging in a containerized environment is a significant challenge, and one which will only be more severe if such an environment is already implemented in an incipient state of the article, when analysis processes are not fully fleshed out.
To alleviate this we consider two possible solutions.
Firstly, high hierarchical granularity, with dedicated containers for individual substeps, as well as with proper logging of outputs at each step, may provide for better debugging capabilities from outside of the container environment.
The strength of this solution is that it implements containers in a coherent fashion to their design and prevalent use in industrial applications, thus minimizing their size and potentially increasing their reusability.
The caveat of this solution is that there is a hard limit to how much debugging can be done from outside of a container environment.
To illustrate, it will always be easier and faster to interactively apply patches or switch dependency versions from inside an environment than from outside of it.
Alternatively, containers could be used as pure virtualization for whatever development environment accompanies the article, cobmined with internal provision of debugging tools.
The strength of this solution is that it seamlessly integrates with development on a fully interactive system and can more easily accommodate novel and exotic workflows.
The drawback of this solution is that it would lead to more bloated and potentially less reusable containers, and it would also require a separate container image re-build once issues have been identified and fixed interactively.

In light of this we would recommend a combination of the above.
Firstly, granular workflow steps using granular container images should be used for all analysis steps which can feasibly be separated and resolved by a general-purpose top-level interface.
Secondly, top-level workflows for which only a bespoke environment should be considered, should be accessible via an interactive target and include basic word processing tools.
An interface of sequential processing steps is available e.g. via the BIDS-App ecosystem, and many tools have containers provided via their authors.
We strongly encourage the use of standard formats such as BIDS for all intermediary steps for which they are applicable, as is done for the interface between preprocessing and subsequent data analysis steps in this study.

Lastly, we highlight the relevance of automated workflows — including both the current reexecution environment workflow and the internal workflow of the original article — for reuse and adaptation.
A key strength of reexecutability is that workflows can easily be derived, with a reliable starting point with respect to successful process execution.
In view of this we recommend a practical approach to the work described herein, whereby the tools can all be considered to be immediately and freely available for inspection and personal use of the reader.

% Mention Gentoo in the context of provenance

%TODO 
% Analogy with a "backup" -- there is no idea if a backup is any good until it is attempted to restore ffrom the backup. The same with studies claiming to be reproducible.


% TODO yoh : I do not understand this (chr) could you please expand this so I can write something more sensible about it?
%      \item Pay attention to user used for invocation of container
%
%        docker -- by default might be root, resultant files would be owned by root, not be able to move etc.
%
%        -e "UID=$(id -u)" -e "GID=$(id -g)"
%
%        and may be `-u` `-g` if used with ``docker''.
%    \end{itemize}

