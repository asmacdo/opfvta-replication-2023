\section{Methods}
% Minutia regarding the infrastructure and resources, versions, where did we run this on, etc.
% Specific usage of the technologies referenced in general terms in Results, e.g. results introduces state-receall via Gentoo, here we say that's `git checkout` and give the hash.
% Specific commands prob go here.
% Describe the containerfile
% pip freeze

%yoh: this could be the section to mention approaches tried but which were not taken to lead to what is presented in the results.
% repronim 5 steps  https://www.repronim.org/5steps.html and
% https://www.sciencedirect.com/science/article/pii/S1053811922007388
%chr: I think that would be better mentioned in the discussion.

\subsection{Data Acquisition}

No new animal data was recorded.
The data forming the substrate for the reproduction analysis was produced by extracting the output \texttt{article.pdf} file from the Novel Worlflow Reexecution.

\subsection{Computing Environments}
Article reexecution was performed on a Debian 6.1.8-1 (2023-01-29) system using the \texttt{x86\_64} architecture, inside containers handled by Podman version \texttt{4.3.1}.
%TODO Things other than Podman we should mention?

\subsection{Data Sources}
The raw data for the article was sourced in BIDS form from Zenodo, an open data repository, via the identifier specified by the original publication \cite{opfvta_bidsdata}.
Mouse brain templates were sourced via a Git repository, “Mouse Brain Templates”, which was updated to allow individual file fetching as part of this study \cite{mbt10}.

%\subsection{YODA}
%yoh: to bind it all up
%TODO: not sure what we should do with YODA here, if it's an established concept, better introduce it in the background, or if we made a contribution by implementing it at the whole-article level, it should go under results.


% TODO(How the data was stored and how we got it)
% yoh: exactly -- let's describe how we acquired'' data from original
% resources: tarballs etc, placed into datalad datasets.  refer where
% possible to the handbook.


%\subsection{Provenance}
%
%datalad run/containers-run
%TODO: not sure what to do with this — is this anything beyond what we cover in Data Sources
