\section{Background}
% Things which should be skippable for anybody familiar with the field.
% Basically just a review of the technologies we build on and extend.
% Long commentary on methods actually goes here.
% Explain stuff like Gentoo or containers here.

\subsection{Reexecutable Research}

%TODO yoh cite hurr-durr reproduction crisis article
Independent verification of published results is a crucial step in the establishing and maintaining trust in shared scientific understanding \cite{rpp}.
The basic feasibility of \textit{de novo} research output generation from the earliest recorded data provenance is known as reexecutability, and  has remained largely unexplored as distinct phenomenon in the broader sphere of research “reproducibility”.
While the scope of \textit{reexecution} is narrower than that of \textit{reproduction}, it constitutes a more well-defined and therefore tractable issue in improving the quality and sustainability of research.
Reexecution is a prerequisite for the reproduction of any complex analysis process, and therefore in its absence reproduction quality assesments become largely intractable.
Further, reexecution constitutes a capability in and of itself, with ample utility in education, training, and resource reuse for novel research purposes (colloquially, “hacking”) — which may accrue even in the absence of accurate result reproduction.

%TODO yoh Is there a review of people sharing their code? If not we can cite a bunch of people who brag about putting their stuff on GH
%TODO asmacdo +1 cool
Free and Open Source Software \cite{foss} has significantly permeated the world of research, and it is presently not uncommon for researchers to publish part of the analysis instructions used in generating published results \cite{TODO} under free and open licenses.
However, such analysis instructions are commonly disconnected from the research output document, which is manually constructed from static inputs.
This precludes automatic reexecution of the full research output, and limits their potential for re-use.
Additionally, without fully reexecutable instructions, data analysis outputs and the positive claims which they support are not verifiably linked to the methods which support them.

“Use cases” annex of the DataLad handbook


In order to optimally leverage extant efforts pertaining to full article reexecution and in order to test reexecutability in the face of high task complexity, we have selected a novel neuroimaging study, identified as OPFVTA based on author naming conventions \cite{opfvta}.
Due to the high task complexity of integrating all data analysis into a coherent and reliable workflow, extant efforts pertaining to full article reexecution are scant and not suitably stress-tested at a larger scale.
One example is a novel neuroimaging study, identified as “OPFVTA” \cite{opfvta} based on author resource naming.
The 2022 article is accompanied by a programmatic workflow via which it can be fully regenerated — based solely on raw data, data analysis instructions, and the natural-language manuscript text — and which is initiated via a simple executable script in the ubiquitous GNU Bash \cite{bash} command language.
The reexecution process in this effort relies on an emerging infrastructure standard, RepSeP \cite{repsep}, which is used by additional other articles, thus providing a larger scope for conclusions that can be drawn from its study.


\subsection{Data Analysis}

One of the hallmarks of scientific data analysis is its intricacy — resulting from the manifold confounds which need to be accounted for, as well as from the breadth of questions which researchers may want to address.
Data analysis can be subdivided into \emph{data preprocessing} and \emph{data evaluation}.
The former consists of data cleaning, reformatting, standardization, and sundry processes which aim to make data suitable for evaluation.
Data evaluation consists of various types of statistical modeling, commonly applied in sequence at various hierarchical steps.

The OPFVTA article, which this study uses as an example, primarily studies effective connectivity, which is resolved via stimulus-evoked neuroimaging analysis.
Stimulus-evoked neuroimaging analysis is one of the more widespread applications, and thus the data analysis workflow (both in terms of \emph{data processing} and \emph{data evaluation}) provides significant analogy to numerous neuroimaging studies.
The data evaluation step for this sort of study is subdivided into “level one” (i.e. within-subject) analysis, and “level two” (i.e. across-subject) analysis, with the results of the latter being further reusable for higher-level analyses \cite{Friston1995}.
In the simplest terms, these modeling steps represent iterative applications of General Linear Modelling (GLM), at increasingly higher orders of abstraction.

% Insert and reference example workflow figure

Computationally, in the case of the OPFVTA article as well as the general case, the various data analysis workflow steps are sharply distinguished by their time cost.
By far the most expensive element is a substage of data preprocessing known as registration.
This commonly relies on iterative gradient descent and can additionally require high-density sampling depending on the feature density of the data.
The second most costly step is the first-level GLM, the cost of which emerges from to the high number of voxels modeled individually for each subject.

The impact of these time costs on reexecution is that rapid-feedback development and debugging can be compromised if the reexecution is monolithic.
While ascertaining the effect of changes in the registration instructions on the final result unavoidably necessitate the reexecution of the entire pipeline — editing natural-language commentary in the article text, or adapting figure styles, should not.
To this end the reference article of this study employs a hierarchical Bash-script structure, consisting of two steps.
The first step, consisting in data preprocessing and all data evaluation steps which operate in voxel space, is handled by one dedicated sub-script.
The second step handles document-specific element generation, i.e. inline statistics, figure, and TeX-based article generation.
The nomenclature to distinguish these two phases introduced by the authors is “high-iteration” and “low-iteration” \cite{repsep}.

Analysis dependency tracking, which is to say monitoring whether files required for the next hierarchical step have changed — and thus whether that step needs to be re-executed — is handled for the high-iteration analysis script via the RepSeP infrastructure, but not for the low-iteration script.


\subsection{Software Dependency Management}

Beyond the hierarchically chained data dependencies, which can be considered internal to the workflow, any data analysis workflow has additional dependencies in the form of software.
This refers to the computational tools called by the workflow — which, given the diversity of research applications, may encompass numerous and complex pieces of software.
Complexity in this sense also refers to the fact that individual software dependencies commonly come with their own software dependencies, which may in turn have further dependencies, and so on.
The resulting network of prerequisites is known as a “dependency graph”, and its resolution is commonly handled by a package manager.

The OPFVTA article in its original form relies on Portage \cite{portage}, the package manager of the Gentoo Linux distribution.
This package manager offers integration across programming languages, source-based package installation, and wide-ranging support for neuroscience software \cite{ng}.
As such, the dependencies of the target article itself are summarized in a standardized format, which is called an ebuild — as if it were any other piece of software.
This format is analogous to the format used to specify dependencies at all further hierarchical levels in the dependency tree.
This affords a homogeneous environment for dependency resolution, as specified by the Package Manager Standard \cite{pms}, which constitutes the authoritative reference for the ebuild format and the behaviour of the package manager given an ebuild.
Additionally, the reference article contextualizes its raw data resource as a dependency, integrating data provision in the same network as software provision.

While the top-level ebuild (i.e. the software dependency requirements of the workflow) is included in the article repository and distributed alongside it, the ebuilds tracking dependencies further down the tree are all distributed via separate repositories.
These repositories are version controlled, meaning that their state at any time point is documented, and they can thus be restored to represent the environment as it would have been generated at any point in the past.


\subsection{Software Dependencies}

The aforementioned infrastructure is relied upon to provide a full set of widely adopted neuroimaging tools, including but not limited to ANTs \cite{ants}, nipype \cite{nipype}, FSL \cite{fsl}, AFNI \cite{afni}, and nilearn \cite{nilearn}.
Additionally, the OPFVTA study employs a higher-level workflow package, SAMRI \cite{samri,irsabi}, which provides workflows optimized for the preprocessing and evaluation of animal neuroimaging data.


\subsection{Containers}

Operating system virtualization is a process whereby an operating system can be emulated inside another running system, the "host", and thus a "guest" environment can be shared with any software and dependencies already installed.
Virtual machines (VMs) are attractive solutions to enable reproducibility first and foremost because the reproducer does can skip the installation of the environment, which is frequently time consuming and requires project-specific knowledge.
Once running, guest machines are self-contained and isolated from the host, which then eliminates the posibility of polluting the host environment.
Perhaps the most important benefit of virtual isolation is significantly improved security that allow modern system adminstrators to safely allow relatively unrestricted usage from semi-trusted users.
Lastly, distributing code in virtual machine images allows the original authors to solve dependency problems that arise from imperfect package managers, imperfect repositories, and constant package updates; instead the authors distribute a locked snapshot of a working system.

System virtualization offers a way to portably freeze and preserve environments, but are limited due to the size of the "full disk images".
Additionally VM's must be be "booted" which can be costly if many instances are needed.
Modern advances in container technology have allowed similar benefits but strip redundancy by making limited use of the host machine, specifically the hypervisor.
Containers enable a complete working environment as small as a few Megabytes, and can be started as quickly as a normal process.
Many container images are publicly available via public image repositories.

Containers technology is not a recent invention, but the term "container" gained popularity alongside the Docker toolset.
Over time Docker and other organizations have come together under a Linux Foundation project, the "Open Container Initiative" (OCI).
The OCI governing body has produced an open specification for containers, which can be used by various container runtimes and toolsets.
OCI complient container images in most cases can be executed identically with Docker, Podman, or other OCI compliant tools.

While OCI images are nearly ubiquitous in the industry, Singularity (recently renamed to Apptainer) is a toolset that was developed specifically for High Performance Computing.
Singularity has support for converting OCI images into singularity images, and recent versions of Apptainer have also added support to natively run OCI containers.
Podman apears to be gaining traction in the HPC community, but Apptainer is still required on many systems.

One of the most significant downsides to using Docker in HPC environments was that it required root privilages.
However, recent advances in container technology have made this unnecessary, and it is now considered best practice to run containers without root privilages when reasonable.

While the original reference OPFVTA article did not leverage this technology, containers can be used improve to improve the reliability and portability of the OPFVTA project.
In this article, the authors will provide a snapshot of OPFVTA functioning at a certain point in time — mitigating process fragility in view of incrementing software dependency versions.
