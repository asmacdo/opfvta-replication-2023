\section{Background}
% Things which should be skippable for anybody familiar with the field.
% Basically just a review of the technologies we build on and extend.
% Long commentary on methods actually goes here.
% Explain stuff like Gentoo or containers here.

\subsection{Reexecutable Research}

%TODO cite hurr-durr reproduction crisis article
While the reproducibility of research findings (i.e. independently verifying a published result or phenomenon) has received considerable attention \supercite{TODO}, reexecutability has remained largely unexplored as a discrete phenomenon.
While the scope of reexecution is narrower, it constitutes a more well-defined and therefore tractable issue in improving the quality and sustainability of research.
On one hand reexecution is a prerequisite for the reproduction of any research which prominently features computational analysis — which encompasses a growing segment of modern experimental research.
On the other hand reexecution constitutes a capability in and of itself, with ample utility in education, training, rapid-feedback development, and resource reuse for novel research purposes (colloquially, “hacking”) — which may accrue even in the absence of reproducibility of accurate reproducibility of results.

%TODO Is there a review of people sharing their code? If not we can cite a bunch of people who brag about putting their stuff on GH
Free and Open Source Software \supercite{foss} has permeated the world of research to a significant extent, and it is presently not uncommon for researchers to freely and or openly publish part of the analysis instructions used in generating published results \supercite{TODO}.
However, such instructions are commonly disconnected from the research output document, which is manually constructed from static inputs.
Consequently, data analysis outputs are not generated in their respective contexts, and therefore disconnected from the positive claims which they support.
This precludes automatic reexecution of the full research output, and limits their potential for re-use.

In order to optimally leverage extant efforts pertaining to full article reexecution and in order to test reexecutability in the face of high task complexity, we have selected a novel neuroimaging study, identified as OPFVTA based on author naming conventions \supercite{opfvta}.
The 2022 article is accompanied by a programmatic workflow via which it can be fully regenerated, based solely on raw data, data analysis instructions, and natural-language commentary — and which is initiated via a simple executable script in the ubiquitous GNU Bash \supercite{bash} command language.
The reexecution process in this effort relies on an emerging infrastructure standard, RepSeP \supercite{repsep}, which is used by additional other articles, thus providing a larger scope for conclusion drawn and standards formulated as part of this study.

% TODO - make proper latex etc:
%  Additional aspects which were not foreseeing in the original execution making it impossible to reexecute to the identical result - no randomization seed was provided or recorded.
%  For the future efforts solutions such as https://github.com/ReproNim/reproseed/ could be used to either inform on the ways to seed specific applications or just to seed them via environment variables.

\subsection{Data Analysis}

One of the hallmarks of scientific data analysis is its intricacy — resulting from the multifaceted confounds which need to be accounted for, as well as from the breadth of questions which extant tools need to be adapted in order to address.
Data analysis, can be subdivided into data preprocessing and data evaluation.
The former consists of data cleaning, reformatting, standardization, and sundry processes required for data to be suitable for evaluation.
Data evaluation consists of various types of modelling, usually applied in sequence at various hierarchical steps.

In the general case of stimulus-evoked neuroimaging analysis, as seen in the OPFVTA article, data evaluation is commonly subdivided into “level one” (i.e. within-subject) analysis, and “level two” (i.e. across-subject) analysis, with the results of the latter being further reusable for higher-level analyses \supercite{Friston1995}.
In effect, these modelling steps usually represent iterative applications of General Linear Modelling (GLM), at increasingly higher orders of abstraction.

Computationally, the various steps of a data analysis pipeline are sharply distinguished by their time cost.
By far the most expensive element in the aforementioned reference pipeline is a substage of data preprocessing known as registration.
This commonly relies on iterative gradient descent and can additionally require high-density sampling depending on the feature density of the data.
The second most costly step is the first-level GLM, the cost of which is mostly due to the high number of voxels modelled individually for each subject.

The impact of these time costs on reexecution is that rapid-feedback development and debugging can be compromised if the reexecution is monolithic.
While ascertaining the effect of changes in the registration instructions on the final result unavoidably necessitate the reexecution of the entire pipeline, editing natural-language commentary in the article text, or adapting figure styles, should not.
To this end the reference article of this study employs a hierarchical Bash-script structure, with data preprocessing and all data evaluation steps which operate in voxel space being handled by a sub-script, and top-level (i.e. inline) statistics, figure, and TeX-based article generation in a separate sub-script.
The nomenclature to distinguish these two phases is “high-iteration” and “low-iteration” \supercite{repsep}.

Analysis dependency tracking, which is to say monitoring whether files required for the next hierarchical step have changed — and thus whether that step needs to be re-executed — is handled for high-iteration analysis script via the RepSeP infrastructure, but not for the low-iteration script.


\subsection{Software Dependency Management}

Beyond the hierarchically evolving data dependencies, which can be considered internal to the workflow, any data analysis workflow has additional dependencies in the form of software.
This refers to the computational tools called by the workflow — and, given the diversity of research applications, may encompass numerous and complex pieces of software.
Complexity in this sense also refers to the fact that individual software dependencies commonly come with their own software dependencies, which may in turn have further dependencies, and so on.
This is known as a dependency graph, which is commonly handled by a package manager.

The OPFVTA article in its original form relies on Portage \supercite{portage}, a package manager characterized by providing integration across programming languages, source-based package installation, and wide-ranging support for neuroscience software \supercite{ng}.
As such, the dependencies of the workflow are summarized in a standardized format, which is called an Ebuild.
This format is analogous to the format used to specify dependencies at all further hierarchical levels in the dependency tree.
This affords a homogeneous environment for dependency resolution, as specified by the Package Manager Standard \supercite{pms}.
Additionally, the reference article contextualizes its raw data origin point as a dependency, providing provenance tracking in the same fashion as for software.

While the top-level Ebuild (i.e. the software dependency requirements of the workflow) is included in the article repository and distributed alongside it, the Ebuilds tracking dependencies further down the tree are all distributed via separate repositories.
These repositories are version controlled, meaning that their state at any time point is documented, and they can thus be restored to represent the environment as it would have been generated at any point in the past.

\subsection{Containers}

Operating system virtualization is a process whereby an operating system can be installed inside another operating system, without being subject to the environment constraints of its parent, and without potentially polluting the parent environment.
Given the complexity of scientific computing environments, such virtualization is attractive on multi-user systems, on systems with lacking package management capabilities, or in instances where the tasks to be executed are fragile and may require bespoke constraints.

One of the prominent instantiations of virtualization are containers, a technology which focuses on portability of operating system images (i.e. containers) across parent operating systems featuring the correct container environment.
Their relevance to open science consists in providing end-users with an accessible environment, which can be ascertained to provide the requirements of a certain top-level workflow, and which does not interfere with their parent environment.

While the reference OPFVTA article does not leverage this technology, containers can improve its portability, as well as provide a snapshot of its functioning at a certain point in time — mitigating process fragility over evolving software dependencies.
